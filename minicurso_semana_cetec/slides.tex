
\setbeamercovered{transparent}
\section{Justificativa}
\subsection{Porquê aprender Python?}
	\begin{frame}{Porquê aprender Python?}
		\begin{columns}
			
			\column{0.33\textwidth}
				\centering
				\begin{block}{\centering Fácil de aprender}
					\justifying
					\tiny 
					
					\begin{itemize}
						\item Intuitivo e fácil
						\item Sintaxe simples
						\item Requer Noções básicas
					\end{itemize}
					
					 
				\end{block}
				
			\column{0.33\textwidth}
				\pause
				
				\begin{block}{\centering Multiplataforma}
					
					\justifying
					\tiny 
					
					\begin{itemize}
						\item Linguagem interpretada
						\item Multiparadigma
						\item Web, mobile ou desktop.
					\end{itemize}
					
				\end{block}
			
			\column{0.33\textwidth}
				\pause
				\begin{block}{\centering Mercado de trabalho}
					
					\justifying
					\tiny
					
					\begin{itemize}
						\item Lider entre as linguagens
						\item Simples e ilimitado
						\item Eficiente e produtivo
					\end{itemize}
					
				\end{block}
					
		\end{columns}
	\end{frame}


\section{Objetivos}
\subsection{Quais os objetivos do curso?}
	\begin{frame}{Objetivos}
		
		\begin{block}{Quais os objetivos do curso?}
			\begin{itemize}[<+->]
				
				\item Oferecer um curso teórico e prático sobre a linguagem Python focada na análise de dados.
				
				\item Apresentar a sintaxe básica da linguagem para quem quer iniciar ou aprimorar o conhecimento na área de Data Science.
				
				\item Abordar algumas ferramentas e técnicas que auxiliam e simplificam a manipulação de dados. 
			\end{itemize}
			
		\end{block}
		
	\end{frame}

\section{Metodologia}
\subsection{Metodologia do curso}
	\begin{frame}{Metodologia}
		\begin{block}{Metodologia do curso}
			\begin{itemize}[<+->]
				\item O curso se dará na apresentação da linguagem Pyhton e sua sintaxe básica, bem como tipos de dados, estruturas de decisão, repetição dentre outras funções internas.
				
				\item Além disso serão aplicados exercícios práticos envolvendo o que foi apresentado.
			\end{itemize}
		\end{block}
	\end{frame}

\section{Conteúdo}
\begin{frame}{Conteúdo}

	\begin{columns}
		\column{0.5\linewidth}
			
			\begin{block}{Tipos de dados}
				
				\begin{itemize}
					\item strings 
					\item inteiros
					\item floats
					\item listas e dicts
				\end{itemize}
			\end{block}
			
			\begin{block}{Estruturas de controle}
				\begin{itemize}
					\item if 
					\item else
					\item elif
				\end{itemize}
			\end{block}
			
				
			
		\column{0.5\linewidth}
			
			\begin{block}{Estruturas de repetição}
				\begin{itemize}
					\item for
					\item while
				\end{itemize}
			\end{block}
		
			\begin{block}{Definição de Funções}
				\begin{itemize}
					\item def
				\end{itemize}
			\end{block}
		
			\begin{block}{Bibliotecas auxiliares}
				\begin{itemize}
					\item Numpy
					\item Matplotlib
					\item Pandas
				\end{itemize}
			\end{block}

	\end{columns}
	

\end{frame}



\section{Conclusão}
